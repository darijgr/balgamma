\documentclass[numbers=enddot,12pt,final,onecolumn,notitlepage]{scrartcl}%
\usepackage[headsepline,footsepline,manualmark]{scrlayer-scrpage}
\usepackage{amssymb}
\usepackage{amsmath}
\usepackage{amsthm}
\usepackage{framed}
\usepackage{comment}
\usepackage{color}
\usepackage[breaklinks=True]{hyperref}
\usepackage[sc]{mathpazo}
\usepackage[T1]{fontenc}
\usepackage{needspace}
\usepackage{tabls}
\usepackage{tikz}
%TCIDATA{OutputFilter=latex2.dll}
%TCIDATA{Version=5.50.0.2960}
%TCIDATA{LastRevised=Saturday, July 12, 2025 20:56:52}
%TCIDATA{SuppressPackageManagement}
%TCIDATA{<META NAME="GraphicsSave" CONTENT="32">}
%TCIDATA{<META NAME="SaveForMode" CONTENT="1">}
%TCIDATA{BibliographyScheme=Manual}
%TCIDATA{Language=American English}
%BeginMSIPreambleData
\providecommand{\U}[1]{\protect\rule{.1in}{.1in}}
%EndMSIPreambleData
\theoremstyle{definition}
\newtheorem{theo}{Theorem}[section]
\newenvironment{theorem}[1][]
{\begin{theo}[#1]\begin{leftbar}}
{\end{leftbar}\end{theo}}
\newtheorem{lem}[theo]{Lemma}
\newenvironment{lemma}[1][]
{\begin{lem}[#1]\begin{leftbar}}
{\end{leftbar}\end{lem}}
\newtheorem{prop}[theo]{Proposition}
\newenvironment{proposition}[1][]
{\begin{prop}[#1]\begin{leftbar}}
{\end{leftbar}\end{prop}}
\newtheorem{defi}[theo]{Definition}
\newenvironment{definition}[1][]
{\begin{defi}[#1]\begin{leftbar}}
{\end{leftbar}\end{defi}}
\newtheorem{remk}[theo]{Remark}
\newenvironment{remark}[1][]
{\begin{remk}[#1]\begin{leftbar}}
{\end{leftbar}\end{remk}}
\newtheorem{coro}[theo]{Corollary}
\newenvironment{corollary}[1][]
{\begin{coro}[#1]\begin{leftbar}}
{\end{leftbar}\end{coro}}
\newtheorem{conv}[theo]{Convention}
\newenvironment{convention}[1][]
{\begin{conv}[#1]\begin{leftbar}}
{\end{leftbar}\end{conv}}
\newtheorem{quest}[theo]{Question}
\newenvironment{question}[1][]
{\begin{quest}[#1]\begin{leftbar}}
{\end{leftbar}\end{quest}}
\newtheorem{warn}[theo]{Warning}
\newenvironment{warning}[1][]
{\begin{warn}[#1]\begin{leftbar}}
{\end{leftbar}\end{warn}}
\newtheorem{conj}[theo]{Conjecture}
\newenvironment{conjecture}[1][]
{\begin{conj}[#1]\begin{leftbar}}
{\end{leftbar}\end{conj}}
\newtheorem{exam}[theo]{Example}
\newenvironment{example}[1][]
{\begin{exam}[#1]\begin{leftbar}}
{\end{leftbar}\end{exam}}
\newenvironment{statement}{\begin{quote}}{\end{quote}}
\newenvironment{fineprint}{\begin{small}}{\end{small}}
\iffalse
\newenvironment{proof}[1][Proof]{\noindent\textbf{#1.} }{\ \rule{0.5em}{0.5em}}
\newenvironment{convention}[1][Convention]{\noindent\textbf{#1.} }{\ \rule{0.5em}{0.5em}}
\newenvironment{question}[1][Question]{\noindent\textbf{#1.} }{\ \rule{0.5em}{0.5em}}
\newenvironment{warning}[1][Warning]{\noindent\textbf{#1.} }{\ \rule{0.5em}{0.5em}}
\fi
\let\sumnonlimits\sum
\let\prodnonlimits\prod
\let\cupnonlimits\bigcup
\let\capnonlimits\bigcap
\renewcommand{\sum}{\sumnonlimits\limits}
\renewcommand{\prod}{\prodnonlimits\limits}
\renewcommand{\bigcup}{\cupnonlimits\limits}
\renewcommand{\bigcap}{\capnonlimits\limits}
\setlength\tablinesep{3pt}
\setlength\arraylinesep{3pt}
\setlength\extrarulesep{3pt}
\setlength\textheight{22.5cm}
\setlength\textwidth{14.8cm}
\newenvironment{verlong}{}{}
\newenvironment{vershort}{}{}
\newenvironment{noncompile}{}{}
\excludecomment{verlong}
\includecomment{vershort}
\excludecomment{noncompile}
\newcommand{\defn}[1]{{\color{darkred}\emph{#1}}}
\newcommand{\RR}{\mathbb{R}}
\newcommand{\QQ}{\mathbb{Q}}
\newcommand{\NN}{\mathbb{N}}
\newcommand{\ZZ}{\mathbb{Z}}
\newcommand{\KK}{\mathbb{K}}
\newcommand{\set}[1]{\left\{ #1 \right\}}
\newcommand{\abs}[1]{\left| #1 \right|}
\newcommand{\tup}[1]{\left( #1 \right)}
\newcommand{\ive}[1]{\left[ #1 \right]}
\newcommand{\floor}[1]{\left\lfloor #1 \right\rfloor}
\newcommand{\mono}{\hookrightarrow}
\newcommand{\epi}{\twoheadrightarrow}
\newcommand{\iso}{\overset{\cong}{\to}}
\newcommand{\symd}{\mathbin{\bigtriangleup}}
\usetikzlibrary{arrows.meta}
\usetikzlibrary{calc}
\usetikzlibrary{chains}
\usetikzlibrary{shapes}
\usetikzlibrary{decorations.pathmorphing}
\usetikzlibrary{lindenmayersystems}
\definecolor{darkgreen}{rgb}{0,.5,0}
\newtheoremstyle{plainsl}
{8pt plus 2pt minus 4pt}
{8pt plus 2pt minus 4pt}
{\slshape}
{0pt}
{\bfseries}
{.}
{5pt plus 1pt minus 1pt}
{}
\theoremstyle{plainsl}
\ihead{An equality for balanced digraphs, version \today}
\ohead{page \thepage}
\cfoot{}
\begin{document}

\title{An equality for balanced digraphs}
\author{Darij Grinberg\thanks{Drexel University, Philadelphia, PA.
\href{mailto:darijgrinberg@gmail.com}{\texttt{darijgrinberg@gmail.com}}},
Benny Liber\thanks{Drexel University, Philadelphia, PA.
\href{mailto:TODO_add_email@gmail.com}{\texttt{TODO\_add\_email@gmail.com}}}}
\date{ROUGH DRAFT, \today}
\maketitle

\begin{abstract}
We prove that ...

\end{abstract}

\section{The theorem}

In this note, we shall discuss \emph{balanced multidigraphs} -- i.e., directed
multigraphs (allowing loops and multiple arcs) in which each vertex satisfies
\textquotedblleft outdegree = indegree\textquotedblright. We recall the
relevant definitions in more detail: A \emph{multidigraph} (henceforth just
\emph{digraph}) means a triple $\left(  V,A,\psi\right)  $, where $V$ and $A$
are two finite sets and $\psi:A\rightarrow V\times V$ is a map. The elements
of $V$ are called the \emph{vertices} of this digraph; the elements $A$ its
\emph{arcs}. The \emph{source} and the \emph{target} of an arc $a\in A$ are
the first and second entries of the pair $\psi\left(  a\right)  $. The
\emph{indegree} $\deg^{-}v$ of a vertex $v\in V$ means the number of arcs
$a\in A$ whose target is $v$. The \emph{outdegree} $\deg^{+}v$ of a vertex
$v\in V$ means the number of arcs $a\in A$ whose source is $v$. We say that a
digraph $\left(  V,A,\psi\right)  $ is \emph{balanced} if and only if each
vertex $v\in V$ satisfies $\deg^{+}v=\deg^{-}v$. For further terminology on
digraphs, we refer to \cite{22s}.\footnote{The famous directed
Euler--Hierholzer theorem (which will not be used in this note) says that a
weakly connected digraph contains an Eulerian circuit if and only if it is
balanced. Thus, weakly connected balanced digraphs are also known as
\emph{Eulerian digraphs}.}

A \emph{to-root} of a digraph $D$ means a vertex $t$ of $D$ such that for each
vertex $v$ of $D$, the digraph $D$ has a path from $v$ to $t$. (Equivalently:
has a walk from $v$ to $t$.)

From now on, we \textbf{fix a balanced digraph }$D=\left(  V,A,\psi\right)  $.
If $B$ is any subset of $A$, then $D\left\langle B\right\rangle $ will denote
the induced subdigraph $\left(  V,B,\psi\mid_{B}\right)  $ of $D$. A subset
$B$ of $A$ will be called \emph{acyclic} if the subdigraph $D\left\langle
B\right\rangle $ has no (directed) cycles.

Given a vertex $s$ of $D$, we define an $s$\emph{-convergence} to be an
acyclic subset $B$ of $A$ such that $s$ is a to-root of the subdigraph
$D\left\langle B\right\rangle $.

We can think of an $s$-convergence as a set $B$ of arcs of $D$ such that by
following the $B$-arcs (i.e., the arcs in $B$) from any vertex $v\in V$, you
will always eventually end up in $s$ (no matter which $B$-arcs you take), and
you will get stuck in $s$.

For any $k\in\mathbb{N}$ and $s\in V$, we let $\gamma_{k}\left(  s\right)  $
denote the number of $s$-convergences of size $k$ (that is, with $k$ arcs).

In this note, we shall prove the following result:

\begin{theorem}
\label{thm.balgamma}This number $\gamma_{k}\left(  s\right)  $ does not depend
on $s$. In other words, $\gamma_{k}\left(  s\right)  =\gamma_{k}\left(
t\right)  $ for any $s,t\in V$.
\end{theorem}

\begin{example}
Let $D$ be the following balanced multidigraph:%
\[%
%TCIMACRO{\TeXButton{tikz multidigraph}{\begin{tikzpicture}[scale=4]
%\begin{scope}[every node/.style={circle,thick,draw=green!60!black}]
%\node(1) at (0,0) {$1$};
%\node(2) at (0,1) {$2$};
%\node(3) at (1,1) {$3$};
%\node(4) at (1,0) {$4$};
%\end{scope}
%\begin{scope}[every edge/.style={draw=black,very thick}, every loop/.style={}]
%\path[->] (1) edge[bend left=20] node[left] {$a$} (2);
%\path[->] (2) edge[bend left=20] node[right] {$b$} (1);
%\path[->] (2) edge node[above] {$c$} (3);
%\path[->] (3) edge node[right] {$d$} (4);
%\path[->] (4) edge node[above] {$e$} (2);
%\path[->] (1) edge[bend left=20] node[above] {$f$} (4);
%\path[->] (4) edge[bend left=20] node[below] {$g$} (1);
%\end{scope}
%\end{tikzpicture}}}%
%BeginExpansion
\begin{tikzpicture}[scale=4]
\begin{scope}[every node/.style={circle,thick,draw=green!60!black}]
\node(1) at (0,0) {$1$};
\node(2) at (0,1) {$2$};
\node(3) at (1,1) {$3$};
\node(4) at (1,0) {$4$};
\end{scope}
\begin{scope}[every edge/.style={draw=black,very thick}, every loop/.style={}]
\path[->] (1) edge[bend left=20] node[left] {$a$} (2);
\path[->] (2) edge[bend left=20] node[right] {$b$} (1);
\path[->] (2) edge node[above] {$c$} (3);
\path[->] (3) edge node[right] {$d$} (4);
\path[->] (4) edge node[above] {$e$} (2);
\path[->] (1) edge[bend left=20] node[above] {$f$} (4);
\path[->] (4) edge[bend left=20] node[below] {$g$} (1);
\end{scope}
\end{tikzpicture}%
%EndExpansion
\]
Then, the $1$-convergences are the subsets%
\[
\left\{  b,d,g\right\}  ,\ \ \ \ \ \ \ \ \ \ \left\{  b,d,e\right\}
,\ \ \ \ \ \ \ \ \ \ \left\{  c,d,g\right\}  ,\ \ \ \ \ \ \ \ \ \ \left\{
b,d,e,g\right\}  ,\ \ \ \ \ \ \ \ \ \ \left\{  b,c,d,g\right\}  .
\]
Hence, $\gamma_{3}\left(  1\right)  =3$ and $\gamma_{4}\left(  1\right)  =2$
and $\gamma_{k}\left(  1\right)  =0$ for all $k\notin\left\{  3,4\right\}  $.
Theorem \ref{thm.balgamma} says that the same numbers appear if you replace
$1$ by any other vertex.

Here is the spanning subdigraph $D\left\langle B\right\rangle $ for
$B=\left\{  b,d,e,g\right\}  $:%
\[%
%TCIMACRO{\TeXButton{tikz multidigraph}{\begin{tikzpicture}[scale=4]
%\begin{scope}[every node/.style={circle,thick,draw=green!60!black}]
%\node(1) at (0,0) {$1$};
%\node(2) at (0,1) {$2$};
%\node(3) at (1,1) {$3$};
%\node(4) at (1,0) {$4$};
%\end{scope}
%\begin{scope}[every edge/.style={draw=black,very thick}, every loop/.style={}]
%\path[->] (2) edge[bend left=20] node[right] {$b$} (1);
%\path[->] (3) edge node[right] {$d$} (4);
%\path[->] (4) edge node[above] {$e$} (2);
%\path[->] (4) edge[bend left=20] node[below] {$g$} (1);
%\end{scope}
%\end{tikzpicture}}}%
%BeginExpansion
\begin{tikzpicture}[scale=4]
\begin{scope}[every node/.style={circle,thick,draw=green!60!black}]
\node(1) at (0,0) {$1$};
\node(2) at (0,1) {$2$};
\node(3) at (1,1) {$3$};
\node(4) at (1,0) {$4$};
\end{scope}
\begin{scope}[every edge/.style={draw=black,very thick}, every loop/.style={}]
\path[->] (2) edge[bend left=20] node[right] {$b$} (1);
\path[->] (3) edge node[right] {$d$} (4);
\path[->] (4) edge node[above] {$e$} (2);
\path[->] (4) edge[bend left=20] node[below] {$g$} (1);
\end{scope}
\end{tikzpicture}%
%EndExpansion
\]

\end{example}

\section{Particular cases}

Theorem \ref{thm.balgamma} was inspired by a talk of Karla Leipold (NORCOM
2025), which made the first author aware of \cite[Lemma 4.1]{LeiVal24}. While
no enumerative questions were discussed in said talk, a scent of bijection was
noticeable. The present note is the result of following this scent.

Some particular cases of Theorem \ref{thm.balgamma} are known:

\begin{enumerate}
\item When $k=\left\vert V\right\vert -1$, the $s$-convergences are precisely
the spanning arborescences of $D$ rooted to $s$. Indeed, the condition
$\left\vert B\right\vert =\left\vert V\right\vert -1$, combined with the
to-rootness of $s$, forces $D\left\langle B\right\rangle $ to be an
arborescence rooted to $s$ (by \cite[Theorem 5.10.5]{22s}). Thus, in the case
$k=\left\vert V\right\vert -1$, Theorem \ref{thm.balgamma} is just
\cite[Corollary 5.12.1]{22s}.

Likewise, if $k<\left\vert V\right\vert -1$, then Theorem \ref{thm.balgamma}
is just saying that $0=0$, since a spanning subdigraph $D\left\langle
B\right\rangle $ with fewer than $\left\vert V\right\vert -1$ arcs cannot have
a to-root.

\item If $D=G^{\operatorname*{bidir}}$ for some (undirected) multigraph
$G=\left(  V,E,\varphi\right)  $, and if $k=\left\vert E\right\vert
=\left\vert A\right\vert /2$, then the $s$-convergences are just the acyclic
orientations of $G$ with unique sink $s$ (why?). Thus, in this case, Theorem
\ref{thm.balgamma} is saying that the number of acyclic orientations of a
given multigraph $G$ with unique sink $s$ does not depend on $s$. This is a
recent result by Foissy \cite[Proposition 4.6]{Foissy22}.

\item I think \cite[Proposition 3.7]{PerPha15} is Theorem \ref{thm.balgamma}
for the maximum possible $k$ for which $\sum_{s\in V}\gamma_{k}\left(
s\right)  \neq0$. (Check this!)
\end{enumerate}

\section{Proof of the theorem}

Our proof of Theorem \ref{thm.balgamma} uses an idea from the proof of
\cite[Proposition 3.7]{PerPha15}.

The main idea is simple: If we decompose $V$ as $V=V_{1}\sqcup V_{2}$ for two
disjoint subsets $V_{1}$ and $V_{2}$, then $\left\vert A\left(  V_{1}%
,V_{2}\right)  \right\vert =\left\vert A\left(  V_{2},V_{1}\right)
\right\vert $, where $A\left(  P,Q\right)  $ denotes the number of arcs in $A$
with source in $P$ and target in $Q$ (see \cite[Exercise 9.1]{22s}). Thus, if
we pick such a decomposition $V=V_{1}\sqcup V_{2}$ and a bijection
$\phi_{V_{1},V_{2}}:A\left(  V_{1},V_{2}\right)  \rightarrow A\left(
V_{2},V_{1}\right)  $, then we can change a subset $B$ of $A$ by applying
$\phi_{V_{1},V_{2}}$ to its subset $B\cap A\left(  V_{1},V_{2}\right)  $. In
other words, we replace $B$ by%
\[
\left(  B\setminus A\left(  V_{1},V_{2}\right)  \right)  \cup\phi_{V_{1}%
,V_{2}}\left(  B\cap A\left(  V_{1},V_{2}\right)  \right)  .
\]
This transformation -- which I will call the $\left(  V_{1},V_{2}\right)
$\emph{-realignment} -- always preserves the acyclicity of $B$ (why?).
However, it can decrease the size of $B$ if $B\cap A\left(  V_{2}%
,V_{1}\right)  \neq\varnothing$.

We can be more strategic about this: Let's say we have an $s$-convergence $B$
and want to make it into a $t$-convergence for another vertex $t$. We let
$V_{1}$ be the set of all vertices $v$ that have a $B$-path to $t$ (that is,
there exists a path from $v$ to $t$ using only edges from $B$), and we let
$V_{2}=V\setminus V_{1}$ be the complement of $V_{1}$ in $V$. Thus, we obtain
a decomposition $V=V_{1}\sqcup V_{2}$ with the property that $B\cap A\left(
V_{2},V_{1}\right)  =\varnothing$ (why?). Therefore, $\left(  V_{1}%
,V_{2}\right)  $-realignment transforms $B$ into a new acyclic subset
$B^{\prime}$ of $A$ with the same size as $B$, but (unless we are in some
degenerate case?) with a larger set of vertices that have a path to $t$ (in
fact, each vertex that had a $B$-path to $t$ will still have a $B^{\prime}%
$-path to $t$, but there will be at least one further vertex with a
$B^{\prime}$-path to $t$).

I think that by repeatedly applying such realignments, we eventually end up
with a $t$-convergence. Thus, we have defined a map from $\left\{
s\text{-convergences}\right\}  $ to $\left\{  t\text{-convergences}\right\}  $.

This map is a bijection! But we can do better:

Let $\mathcal{P}_{k}\left(  A\right)  $ denote the set of all $k$-element
subsets of $A$. Fix two vertices $s\neq t$ of $D$ (it is clear that
$\gamma_{k}\left(  s\right)  =\gamma_{k}\left(  t\right)  $ holds if $s=t$),
and define
\begin{align*}
\Gamma_{k}\left(  s\right)  :=  &  \ \left\{  B\in\mathcal{P}_{k}\left(
A\right)  \text{ is acyclic}\ \mid\ \text{each vertex can }B\text{-reach
}s\right\} \\
=  &  \ \left\{  B\in\mathcal{P}_{k}\left(  A\right)  \ \mid\ B\text{ is an
}s\text{-convergence}\right\}  ;\\
\Gamma_{k}\left(  t\right)  :=  &  \ \left\{  B\in\mathcal{P}_{k}\left(
A\right)  \text{ is acyclic}\ \mid\ \text{each vertex can }B\text{-reach
}t\right\} \\
=  &  \ \left\{  B\in\mathcal{P}_{k}\left(  A\right)  \ \mid\ B\text{ is a
}t\text{-convergence}\right\}  ;\\
\Gamma_{k}\left(  s,t\right)  :=  &  \ \left\{  B\in\mathcal{P}_{k}\left(
A\right)  \text{ is acyclic}\ \mid\ \text{each vertex can }B\text{-reach
}s\text{ or }t\right\}  ,
\end{align*}
where \textquotedblleft$B$-reach\textquotedblright\ means \textquotedblleft
reach through a path of $D\left\langle B\right\rangle $\textquotedblright.
Clearly, $\Gamma_{k}\left(  s,t\right)  $ contains both $\Gamma_{k}\left(
s\right)  $ and $\Gamma_{k}\left(  t\right)  $ as subsets. Thus,
\[
\left\vert \Gamma_{k}\left(  s,t\right)  \setminus\Gamma_{k}\left(  t\right)
\right\vert =\left\vert \Gamma_{k}\left(  s,t\right)  \right\vert
-\underbrace{\left\vert \Gamma_{k}\left(  t\right)  \right\vert }_{=\gamma
_{k}\left(  t\right)  }=\left\vert \Gamma_{k}\left(  s,t\right)  \right\vert
-\gamma_{k}\left(  t\right)
\]
and likewise $\left\vert \Gamma_{k}\left(  s,t\right)  \setminus\Gamma
_{k}\left(  s\right)  \right\vert =\left\vert \Gamma_{k}\left(  s,t\right)
\right\vert -\gamma_{k}\left(  s\right)  $. Thus, in order to prove that
$\gamma_{k}\left(  s\right)  =\gamma_{k}\left(  t\right)  $, it suffices to
construct a bijection $\Gamma_{k}\left(  s,t\right)  \setminus\Gamma
_{k}\left(  t\right)  \rightarrow\Gamma_{k}\left(  s,t\right)  \setminus
\Gamma_{k}\left(  s\right)  $.

To do so, we proceed as follows:

\begin{itemize}
\item Define a map $f:\Gamma_{k}\left(  s,t\right)  \setminus\Gamma_{k}\left(
t\right)  \rightarrow\Gamma_{k}\left(  s,t\right)  \setminus\Gamma_{k}\left(
s\right)  $ as follows: Let $B\in\Gamma_{k}\left(  s,t\right)  \setminus
\Gamma_{k}\left(  t\right)  $. Set%
\begin{align*}
V_{1}  &  =\left\{  \text{vertices that can }B\text{-reach }t\right\}  ;\\
V_{2}  &  =V\setminus V_{1}.
\end{align*}
Then, let $f\left(  B\right)  :=\phi_{V_{1},V_{2}}\left(  B\right)  $.

\item Define a map $g:\Gamma_{k}\left(  s,t\right)  \setminus\Gamma_{k}\left(
s\right)  \rightarrow\Gamma_{k}\left(  s,t\right)  \setminus\Gamma_{k}\left(
t\right)  $ as follows: Let $B\in\Gamma_{k}\left(  s,t\right)  \setminus
\Gamma_{k}\left(  s\right)  $. Set%
\begin{align*}
W_{2}  &  =\left\{  \text{vertices that can }B\text{-reach }s\right\}  ;\\
W_{1}  &  =V\setminus W_{2}.
\end{align*}
Then, let $g\left(  B\right)  :=\phi_{W_{1},W_{2}}^{-1}\left(  B\right)  $.
\end{itemize}

We must show that $f\circ g=\operatorname*{id}$ and $g\circ
f=\operatorname*{id}$. Obviously, the main step for proving $g\circ
f=\operatorname*{id}$ is to prove that when we apply $g$ to $f\left(
B\right)  $, then the $W_{1}$ and $W_{2}$ constructed will be precisely the
$V_{1}$ and $V_{2}$ that were used in the application of $f$ to $B$. The proof
of $f\circ g=\operatorname*{id}$ can be avoided using the pigeonhole
principle, but it might also just be analogous (note that $\phi_{V_{2},V_{1}%
}^{-1}$ is \textquotedblleft another option\textquotedblright\ for
$\phi_{V_{1},V_{2}}$).

\section{Further remarks on $s$-convergences}

\begin{itemize}
\item It is easy to see that an $s$-convergence contains no arcs with source
$s$.

\item An equivalent way to define $s$-convergences is this: An $s$-convergence
is an acyclic subset $B$ of $A$ such that $s$ is the only sink (= vertex with
no outgoing arcs) of the subdigraph $D\left\langle B\right\rangle $.
\end{itemize}

\begin{noncompile}
In proving Conjecture \ref{conj.karla}, we can WLOG assume that $D$ is weakly
connected (otherwise, $\gamma_{k}\left(  s\right)  =0$ for all $k$ and $s$).
In this case, $D$ has a Eulerian circuit (\cite[Theorem 4.7.2]{22s}), hence is
strongly connected. Thus, in proving $\gamma_{k}\left(  s\right)  =\gamma
_{k}\left(  t\right)  $, we can WLOG assume that $D$ has an arc from $s$ to
$t$. Not sure if this makes things any simpler. At least it might make it
simpler to check random $D$'s, since we can sample $D$ by randomly drawing an
Eulerian circuit.

Another simplification we can make: If $v$ is a vertex with $\deg^{-}%
v=\deg^{+}v=1$, then we can replace the arcs $p\rightarrow v$ and
$v\rightarrow q$ by a single arc $p\rightarrow q$. Thus, WLOG we can assume
that each vertex $v$ of $D$ has $\deg^{-}v=\deg^{+}v\geq2$.
\end{noncompile}

\begin{thebibliography}{99999999}                                                                                         %


\bibitem[22s]{22s}\href{https://arxiv.org/abs/2308.04512v3}{Darij Grinberg,
\textit{An introduction to graph theory}, arXiv:2308.04512v3.}

\bibitem[Foissy22]{Foissy22}%
\href{https://arxiv.org/abs/2201.11974v5}{Lo\"{\i}c Foissy, \textit{Bialgebras
in cointeraction, the antipode and the eulerian idempotent},
arXiv:2201.11974v5.}

\bibitem[LeiVal24]{LeiVal24}\href{https://arxiv.org/abs/2402.09914v3}{Karla
Leipold, Frank Vallentin, \textit{Computing the EHZ capacity is NP-hard},
arXiv:2402.09914v3.}

\bibitem[PerPha15]{PerPha15}K\'{e}vin Perrot, Trung Van Pham, \textit{Feedback
Arc Set Problem and NP-Hardness of Minimum Recurrent Configuration Problem of
Chip-Firing Game on Directed Graphs}, Annals of Combinatorics \textbf{19}
(2015), pages 373--396.
\end{thebibliography}


\end{document}